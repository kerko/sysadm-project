\documentclass[a4paper,11pt,pdftex,english]{article}
\renewcommand{\arraystretch}{1.5}
\usepackage{babel}
\usepackage{amsmath}
\usepackage{graphicx}
\usepackage[utf8]{inputenc}
\usepackage{epsfig}
\usepackage{graphics}
\usepackage{palatino}
\renewcommand{\ttdefault}{lmtt}
\usepackage{enumerate}
\usepackage{mdwlist}
\usepackage{geometry}
%\usepackage{textcomp}
%\usepackage{type1cm}
\usepackage[table]{xcolor}
\usepackage{varioref}
\usepackage{url}
\usepackage[bookmarks=true, linkcolor=blue,
citecolor=blue,urlcolor=blue,colorli nks=true, breaklinks=true,
pagebackref=true, hyperindex=true,bookmarksopen=true]{hyperref}
\usepackage{enumitem}
\newlist{arrowlist}{itemize}{1}
\setlist[arrowlist]{label=$\Rightarrow$}

\usepackage[acronym, section, nonumberlist, toc]{glossaries}
\usepackage{minted}
\usemintedstyle{emacs}


\frenchspacing

%% Block style paragrphs
\setlength\parskip{\medskipamount}
\setlength\parindent{0pt}

\makeatletter
\renewcommand{\topfraction}{.9}
\renewcommand{\bottomfraction}{.8}
\renewcommand{\textfraction}{.15}
\renewcommand{\floatpagefraction}{.66}
\renewcommand{\dbltopfraction}{.66}
\renewcommand{\dblfloatpagefraction}{.66}
\setcounter{topnumber}{9}
\setcounter{bottomnumber}{9}
\setcounter{totalnumber}{20}
\setcounter{dbltopnumber}{9}

\makeatother

\makeglossaries

%% Glossary Section
%%% Abreviations
\newacronym{VM}{VM}{Virtual Machine}

 

%%% Glossary Terms
\newglossaryentry{box}
{
  name=box,
  description={is a base virtual machine used as the base for setting up new virtual machines, also called "base box"},
   plural=boxes
}
\newglossaryentry{provisioner}
{
 name=provisioner,
  description={is a tool used to configure a virtual machine, so it gets into the state you want it to have. This includes installing software and changing configurations}
}
\newglossaryentry{provider}
{
 name=provider,
  description={the program used by Vagrant to run the virtual machine, several options are available here}
}
\newglossaryentry{repository}
{
 name=repository,
  description={a directory that contains software packages which can be installed on the system using a package manager}
}






\title{Building Development Environments using Docker and Vagrant }
\author{Urs Oberdorf (141627) and Jan Kerkenhoff (141628)}

\begin{document}

\maketitle

\abstract{Abstract needs to be entered here}

\thispagestyle{empty}

\clearpage
\pagenumbering{roman}
\setcounter{page}{1}
\tableofcontents

\clearpage
\pagenumbering{arabic}

\section{Introduction}

One of the most annoying bug’s while developing are always the “works on my machine” bugs, you spent numerous hours troubleshooting some developers environmental problems with mostly no added value to you product. But isn’t there an easier way to handle this? There is, using \glsreset{VM}to give every developer the same environment just the way you want it. But simply using virtual machine images produces a lot of overhead each time changes to the environment need to be made. Every change requires sending out several hundreds megabytes to every developer, while this can be done at a reasonable speed via internal networks, you are out of luck if your project span the entire globe and developers do not have access to Gigabit speeds anymore. To overcome this additional flaw of \glspl{VM} we set out to find a way to create and agile development environment workflow that does not require huge amounts of data to be send for every change.
We finally settled on a combination \glspl{VM} managed by Vagrant and Puppet as \gls{provisioner} using Virtual Box as the \gls{provider} and additionally Docker to handle the application environment inside the \gls{VM} . While our system does not get rid of downloading a base \gls{VM} image and Docker containers, this step is only required once.


\section{SLA}

Our plan is to show how to use Vagrant to set up consistent development environments. We will show this for the case of web development. We also combine Vagrant with Docker to create a more production like environment, with separate instances and IP's for webserver, database and other services. Using Docker we minimize the performance impact of each development environment. Since we run Docker containers instead of additional \gls{VM}'s for web server and databases, we save valuable resources on the developers system. Because Docker containers share resource with the host operating system they run on, it is possible to run more than one development environment at the same time. Also Vagrant reduces the effort to get the whole environment up and running to a single Vagrant up .

We will provide a development environment with the following software:

\begin{itemize}
\item Chrome, Opera and Firefox
\item git
\item Atom as a graphical editor
\item optional Ruby and/or Python
\item Wireshark for debugging
\item additional tools needed for web development
\end{itemize}
Following Docker containers will also run in the Vagrant provided \gls{VM}:
\begin{itemize}
\item Webserver (apache)
\item Database (mysql)
\end{itemize}

\section{Survey of similar projects}

We looked similar projects and while we found many projects that setup development environments using Vagrant and Puppet, one of them used Vagrant to create the \gls{VM} your actutally developing in. The normal approach with Vagrant is to use to create your production environment and then develop on your local machine. There exists some projects which aim to setup the local developer machines, most promising looking are Boxeen by Github for Mac OS X and Boxstarter for Windows. 
There are even web services aimed at creating environments using Puppet and Vagrant, most notably is PuPHPet, which offers a sleek web GUI to setup web server images the way you want them to be.

\section{Actual Project}

\subsection{Creating a base Image for Vagrant}
Vagrant uses so called “\glspl{box}”, which are basically \gls{VM} images which are used as a base for the \gls{VM} Vagrant creates, to setup the environment. There are already pre build \glspl{box} available for almost all operating systems, but they are all designed to be run headless and support no Graphical Interface. Since we wanted to go one step further and use Vagrant to create a virtual machine in which our developers would work, we need to create our own Ubuntu \gls{box}. Creating a \glspl{box} with Vagrant is really simple you just to the following steps:
\begin{itemize}
\item Install your Operating system of choice in the \gls{provider} of choice ( Ubuntu and Virtual Box in our case)
\item Install Puppet and every other thing you need to provision your \gls{VM} (eg. Virtual\gls{box} guest additions)
\item Add Vagrant user and copy over Vagrant SSH keys and enable password less sudo
\item run \verb|Vagrant package --base my-virtual-machine}| to let Vagrant pack the \gls{VM} into a baseball
\end{itemize}
Full guide can be found here : \url{https://docs.Vagrantup.com/v2/virtualbox/boxes.html}



\subsection{Creating the environment using Vagrant}

Vagrant uses our previous created \gls{box} to set up the Ubuntu \gls{VM} with everything we want to include.  When we call Vagrant up it imports the specified base \gls{box} and starts it, Vagrant then also takes care of mounting shared folder inside the \gls{VM}, some of them are specified and some are automatically mounted like the ones needed for the Puppet run. When the machine successfully started, Puppet is run with our manifest to provision the \gls{VM} with everything we want. In order to do this, Puppet installs the repositories needed for the browsers. Afterwards Puppet installs the packages for the browsers and tools from the repositories. After that Puppet takes care of Docker, by installing the package and downloading the Docker container images onto the host. Afterwards Puppet runs the Docker containers and the application runtime , in our case a simple html+php+mysql web server, is ready to accept connections.

\subsection{Using Docker to provide application runtime environment inside the \gls{VM}}




\section{Results and discussion}

This is the longest chapter, feel free to include some figures. Include
discussion of all problems you have encountered.



\section{Conclusions}

So what is the punchline, does it work or not? Do you recommend doing this
for others?
\cleardoublepage{}
\section{Appendix}
\subsection{Vagrantfile}
\inputminted[linenos=true]{ruby}{../../vagrant/Vagrantfile}
\subsection{Puppetfiles}
We used the  \verb|default.pp| as the main Puppet file run by Vagrant. Also we created some very simple modules for some thing like the bowsers that required more than a few Puppet statements. Our self created modules are attached here and we also added a list of all modules from the puppet forge we used.
\subsubsection{default.pp}
\inputminted[linenos=true]{puppet}{../../puppet/manifests/default.pp}
\subsubsection{module manifests}
Manifest to install googlechrome:
\inputminted[linenos=true]{puppet}{../../puppet/modules/googlechrome/manifests/init.pp}
Manifest to install opera:
\inputminted[linenos=true]{puppet}{../../puppet/modules/opera/manifests/init.pp}

List of modules from the puppet forge: \\
\begin{tabular}{l r}
Docker &  \url{https://forge.puppetlabs.com/garethr/docker} \\

\end{tabular}

\cleardoublepage{}
\printglossary[type=\acronymtype,title=Acronyms,style=long]

\printglossary[style=altlist,title=Glossary]

\bibliographystyle{acmdoi}
\bibliography{template}

\end{document}
